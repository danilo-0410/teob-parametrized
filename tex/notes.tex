\documentclass[prd,amssymb,amsmath,amsfonts,nofootinbib,reprint,showpacs,longbibliography]{revtex4-1}

\usepackage{graphicx}
\usepackage{lmodern}
\usepackage{amsmath,amssymb}
\usepackage{mathrsfs}
\usepackage{amsfonts}
\usepackage[utf8]{inputenc}
\usepackage{url}
\usepackage[colorlinks]{hyperref}
\usepackage[table]{xcolor}
\usepackage{multirow}
\usepackage[normalem]{ulem}
\usepackage{lipsum}

\usepackage{marvosym}
\usepackage{enumerate}
\usepackage{color,soul}
\usepackage{acronym}

\usepackage{bm}
\usepackage{color}
\usepackage{commath}
\allowdisplaybreaks
\usepackage{multirow}

\def\lm{{\ell m}}
\newcommand{\avg}[1]{\rangle{#1}\langle}
\newcommand{\scri}{{\mathrsfs{I}}}
\newcommand{\stf}[1]{{\langle {#1} \rangle}}
\newcommand{\ord}{\mathcal{O}}
\newcommand{\f}{\frac}
\newcommand{\gsf}{\text{1GSF}}
\newcommand{\bk}{\text{0GSF}}
\newcommand{\el}{\ell}
\newcommand{\nn}{\nonumber}
\newcommand{\mbf}[1]{\mathbf{#1}}
\newcommand{\LOhat}{\!\hat{\,\mathbf{L}}_0}
\newcommand{\Lhat}{\!\hat{\,\mathbf{L}}_\text{N}}
\newcommand{\Jhat}{\hat{\mathbf{J}}_\text{N}}
\newcommand{\Sa}{\mbf{S}_1}
\newcommand{\Sb}{\mbf{S}_2}
\newcommand{\hs}{\hat{s}}
\newcommand{\hS}{\hat{S}}
\newcommand{\Lhatdot}{\dot{\hat{\,\mathbf{L}}}_\text{N}}
\newcommand{\LhatdotNew}{\dot{\hat{\,\mathbf{L}}}_{\text{N},\perp}}
\newcommand{\Sadot}{\dot{\mbf{S}}_1}
\newcommand{\Sbdot}{\dot{\mbf{S}}_2}
\newcommand{\Jdot}{\dot{\mbf{J}}}
\newcommand{\rdot}{\dot{\mbf{r}}}
\newcommand{\vdot}{\dot{\mbf{v}}}
\newcommand{\ud}{\mathrm{d}}
\newcommand{\Ldot}{\dot{\mbf{L}}_{\text N}}
\newcommand{\LN}{\mbf{L}_{\text N}}
\newcommand{\nnm}{\nonumber}
\newcommand{\mce}{\mathcal{E}}
\newcommand{\mcb}{\mathcal{B}}

\def\TEOBResumS{\texttt{TEOBResumS}}
\def\TEOBResumSGIOTTO{\texttt{TEOBResumSGIOTTO}}
\def\TEOBResumSecce{\texttt{TEOBResumSecce}}
\def\bajes{\texttt{bajes}}
\def\Boldtheta{\boldsymbol{\theta}}
\def\Boldd{\textbf{d}}
\def\TEOBResumSDali{\texttt{TEOBResumS-Dalí}}
\def\TEOBB{\texttt{Beyond-TEOBResumS}}
\def\c3{c_{\rm{N}^3\rm{LO}}}
\def\mbhf{M_{\rm BH}^f}
\def\abhf{a_{\rm BH}^f}
\def\alphalm0{\alpha_{\ell m 0}}
\def\omegalm0{\omega_{\ell m 0}}


% Define capitalized acronyms
\newacro{adm}[ADM]{Arnowitt-Deser-Misner}
\newacro{bbh}[BBH]{binary black hole}
% \newacroplural{bbh}[BBHs]{binary black holes}
\newacro{bh}[BH]{black hole}
\newacroplural{bh}[BHs]{black holes}
\newacro{bhns}[BHNS]{black hole-neutron star}
\newacro{bhpt}[BHPT]{black hole perturbation theory}
\newacro{bns}[BNS]{binary neutron star}
\newacro{bf}[BF]{Bayes' factor}
\newacro{cbc}[CBC]{compact binary coalescence}
\newacro{ce}[CE]{Cosmic Explorer}
\newacro{da}[DA]{data analysis}
\newacro{et}[ET]{Einstein Telescope}
\newacro{eob}[EOB]{Effective-One-Body}
\newacro{eom}[EOM]{equations of motion}
\newacro{fd}[FD]{frequency domain}
\newacro{fft}[FFT]{Fast Fourier transform}
\newacro{gw}[GW]{gravitational-wave}
\newacroplural{gw}[GWs]{Gravitational-waves}
\newacro{gr}[GR]{general relativity}
\newacro{grb}[GRB]{gamma-ray burst}
\newacro{grhd}[GRHD]{general-relativistic hydrodynamics}
\newacro{gwosc}[GWOSC]{Gravitational Wave Open Science Center}
\newacro{gwtc1}[GWTC-1]{the first gravitational-wave transients catalog}
\newacro{gsf}[GSF]{Gravitational Self Force}
\newacro{hm}[HM]{Higher mode}
\newacroplural{hm}[HMs]{Higher modes}
\newacro{ifo}[IFO]{interferometer}
\newacro{imr}[IMR]{inspiral-merger-ringdown}
\newacro{im}[IMR]{inspiral-to-merger}
\newacro{kagra}[KAGRA]{Kamioka Gravitational Wave Detector}
\newacro{ligo}[LIGO]{Laser Interferometer Gravitational-Wave Observatory}
\newacro{lisa}[LISA]{Laser Interferometer Space Antenna}
\newacro{lr}[LR]{Light Ring}
\newacro{lso}[LSO]{Last Stable Orbit}
\newacro{lvc}[LVC]{LIGO-Virgo Collaboration}
\newacro{lvk}[LVK]{LIGO-Virgo-Kagra Collaboration}
\newacro{lo}[LO]{leading order}
\newacro{ns}[NS]{neutron star}
\newacroplural{ns}[NSs]{neutron stars}
\newacro{nr}[NR]{numerical relativity}
\newacro{nqc}[NQC]{Next-to-quasicircular corrections}
\newacro{nlo}[NLO]{next-to-leading order}
\newacro{nnlo}[NNLO]{next-to-next-to-leading order}
\newacro{n3lo}[N3LO]{next-to-next-to-next-to-leading order}
\newacro{n4lo}[N3LO]{next-to-next-to-next-to-next-to-leading order}
\newacro{ode}[ODE]{Ordinary Differential Equation}
\newacroplural{ode}[ODEs]{Ordinary Differential Equations}
\newacro{pn}[PN]{post-Newtonian}
\newacro{pm}[PM]{post-Minkowskian}
\newacro{pe}[PE]{parameter estimation}
\newacro{psd}[PSD]{power spectral density}
\newacro{pa}[PA]{post-adiabatic}
\newacro{qnm}[QNM]{quasi-normal mode}
\newacro{qc}[QC]{quasi-circular}
\newacro{snr}[SNR]{signal-to-noise ratio}
\newacro{spa}[SPA]{stationary-phase approximation}
\newacro{sxs}[SXS]{Simulating eXtreme Spacetimes}
\newacro{td}[TD]{time domain}
\newacro{ng}[NG]{Nect Generation}


\definecolor{cyan}{rgb}{0,0.9,0.9}
\definecolor{orange}{rgb}{0.9,0.5,0}
\definecolor{magenta}{rgb}{1,0,1}
\definecolor{purple}{rgb}{0.8,0.4,0.8}
\definecolor{gray}{rgb}{0.8242,0.8242,0.8242}
\definecolor{dodgerblue}{rgb}{0.12, 0.56, 1.0}

\newcommand{\RG}[1]{{\textcolor{dodgerblue}{{RG: #1}} }}
\newcommand{\SN}[1]{{\textcolor{purple}{{SN: #1}} }}
\newcommand{\DC}[1]{{\textcolor{orange}{{DC: #1}} }}


\newcommand{\todo}[1]{\textcolor{orange}{\texttt{TODO: #1}}} 
\newcommand{\red}[1]{\textcolor{red}{#1}} 
\newcommand{\bl}[1]{\textcolor{blue}{#1}} 
\newcommand{\cor}[2]{\sout{#1}\textcolor{red}{#2}} 
\newcommand{\old}[1]{\textcolor{gray}{\sout{#1}}}
\newcommand{\new}[1]{\red{#1}}

\newcommand{\eo}[0]{\hat{E}_0}
\newcommand{\lo}[0]{\hat{L}_0}
\newcommand{\bphi}[0]{\bar{\varphi}}
\newcommand{\dbphi}[0]{\dot{\bar{\varphi}}}
\newcommand{\dphi}[0]{\dot{\varphi}}

\interfootnotelinepenalty=10000

\begin{document}

\title{Beyond-TEOBResumS}
% \author{Danilo \surname{Chiaramello}${}^{1,2}$}
% \author{Rossella \surname{Gamba}${}^{3,4}$}

% \affiliation{${}^{1}$Dipartimento di Fisica, Universit\`a di Torino, Via P. Giuria 1, 10125 Torino, Italy}
% \affiliation{${}^{2}$ INFN Sezione di Torino, Torino, 10125, Italy}
% \affiliation{${}^{3}$ Institute for Gravitation \& the Cosmos, The Pennsylvania State University, University Park PA 16802, USA}
% \affiliation{${}^{4}$ Department of Physics, University of California, Berkeley, CA 94720, USA}

\begin{abstract}
Notes on the implementation and testing of \TEOBB, a version of the \TEOBResumSDali~model that includes
deviations from a set of NR-informed quantities that have a role in waveform generation.
\end{abstract}

\date{\today}
\maketitle

% reset all acronyms
\acresetall

\section{Introduction}

\section{The model}

We build upon the most recent version of the generic \ac{bbh} model \TEOBResumSDali, incorporating non-circular
orbits and spin precession~\cite{Nagar:2024oyk, Gamba:2024cvy}. The parametrized model is constructed by
allowing user-input deviations from a few of the NR-informed quantities that enter the dynamics or waveform
template. Specifically, regarding the dynamical sector and the inspiral, we consider deviations from two
NR-calibrated parameters: $a_6^c$, an effective 6\ac{pn} coefficient entering the (resummed) gravitational
potential $A(r)$, and $c_{\rm{N}^3\rm{LO}}$, a next-to-next-to-next-to-leading-order term in the spin-orbit
part of the Hamiltonian. We use in this case additive deviations from the fitted values:
\begin{subequations}
\begin{align}
a_6^c &\rightarrow a_6^c + \delta a_6^c \\
\c3 &\rightarrow \c3 + \delta \c3 \, .
\end{align}
\end{subequations}
Concerning the waveform model, the deformations we allow are the following:
\begin{itemize}
\item Fractional deviations from the mass and spin of the final remnant \ac{bh}, which determine the properties
of the post-merger signal:
\begin{subequations}
\begin{align}
\mbhf &\rightarrow \mbhf (1 + \delta \mbhf) \\
\abhf &\rightarrow \abhf (1 + \delta \abhf)
\end{align}
\end{subequations}
The bare values used in the standard version of the model are highly accurate fits of NR data~\cite{paper}.
%
\item Deviations from the NR-fitted values of the fundamental QNM frequencies $\sigma_{\ell m 0} = \alphalm0
 + i \omegalm0$ of each mode. We separately allow fractional deformations of the inverse damping time,
 $\alphalm0 = \tau_{\ell m 0}^{-1}$, and the oscillatory part of the frequency, $\omegalm0$:
\begin{subequations}
\begin{align}
\alphalm0 &\rightarrow \alphalm0 (1 + \delta \alphalm0) \\
\omegalm0 &\rightarrow \omegalm0 (1 + \delta \omegalm0)
\end{align}
\end{subequations}
We require that the damping time deviations satisfy $\delta \alphalm0 > -1$, to avoid exponentially growing
post-merger signals.
%
\item For each mode, we consider deviations from NR-fitted values of the amplitude and frequency of the
waveform at merger, defined as the peak of the (co-precessing) $(2,2)$ mode. If a waveform mode is decomposed
as $h_{\ell m} = A_{\ell m} e^{-i \phi_{\ell m}}$ and $\omega_{\ell m} = d\phi_{\ell m}/dt$,
\begin{subequations}
\begin{align}
A_{\ell m}^{\rm mrg} &\rightarrow A_{\ell m}^{\rm mrg} (1 + \delta A_{\ell m}^{\rm mrg}) \\
\omega_{\ell m}^{\rm mrg} &\rightarrow \omega_{\ell m}^{\rm mrg} (1 + \delta \omega_{\ell m}^{\rm mrg})
\end{align}
\end{subequations}
%
\item For each mode, we also separately implement deviations from waveform properties measured at the \ac{nqc}
extraction point, $t_{\ell m}^{\rm NQC}$. We again use fractional deviations, so that:
\begin{subequations}
\begin{align}
x_{\ell m}^{\rm NQC} &\rightarrow x_{\ell m}^{\rm NQC} (1 + \delta x_{\ell m}^{\rm NQC}) \\
x &\in \{A, \dot{A}, \omega, \dot{\omega}\} \, .
\end{align}
\end{subequations}
\end{itemize}
%
A note on the NQC point deviations: currently the location of the \ac{nqc} point in \TEOBResumSDali~differs between modes. 
We have three different implementations: 
\begin{itemize}

\item[(i)] For the $(2,2), (3,2), (4,2)$ and $(4,3)$ modes, the \ac{nqc} point is tied to the peak time of the $(2,2)$
mode amplitude of the EOB waveform, $t_{\ell m}^{\rm NQC} = t_{A_{22}^{\rm peak}}^{EOB} + 2 + \Delta t_{\ell m}$, 
where $t_{A_{22}^{\rm peak}}^{EOB} = t_{\Omega_{\rm orb}^{\rm peak}} - 2 - \Delta t_{\rm NQC}$, $t_{\Omega_{\rm orb}^{\rm peak}}$
is the time when the \textit{pure} orbital frequency peaks, $\Delta t_{\rm NQC} = 1$, and $\Delta t_{\ell m}$ is an NR-fitted parameter
encoding the delay between the peak of a generic mode $(\ell, m)$ with respect to the $(2,2)$ mode.
The NQC extraction point thus falls in the post-merger part of the waveform, where the ringdown
template is valid; so, the amplitude, frequency and their derivatives can be computed by evaluating the
template at the appropriate time. This means that \textit{the merger time deviations also influence the NQC time
quantities}.
\item[(ii)] The \ac{nqc} point is located as defined above, but the \ac{nqc} quantities
are computed through direct NR fits; in this case, the merger and NQC time deviations are completely independent.
\item[(iii)] For the $(2,1), (3,3)$ and $(4,4)$ modes, the NQC point coincides
with the peak of the $(2,2)$ mode, $t_{\ell m}^{\rm NQC} = t_{A_{22}^{\rm peak}}^{\rm EOB}$; direct NR fits 
are used in these cases as well. Here, the merger and NQC amplitude and frequency deviations actually affect
the same physical quantities, while acting independently and entering the model at (slightly) different stages.
\end{itemize}
\DC{Check all this}

\bibliography{refs, local}

\end{document}
